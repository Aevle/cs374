% ---------
%  Compile with "pdflatex hw0".
% --------
%!TEX TS-program = pdflatex
%!TEX encoding = UTF-8 Unicode

\documentclass[11pt]{article}
\usepackage{jeffe,handout,graphicx}
\usepackage[utf8]{inputenc}		% Allow some non-ASCII Unicode in source

%  Redefine suits
\usepackage{pifont}
\def\Spade{\text{\ding{171}}}
\def\Heart{\text{\textcolor{Red}{\ding{170}}}}
\def\Diamond{\text{\textcolor{Red}{\ding{169}}}}
\def\Club{\text{\ding{168}}}

\def\Cdot{\mathbin{\text{\normalfont \textbullet}}}
\def\Sym#1{\textbf{\texttt{\color{BrickRed}#1}}}



% =====================================================
%   Define common stuff for solution headers
% =====================================================
\Class{CS/ECE 374}
\Semester{Spring 2017}
\Authors{1}
\AuthorOne{Birgit Alitz}{alitz2}
%\Section{}


% =====================================================
\begin{document}

% ---------------------------------------------------------


\HomeworkHeader{0}{1}	% homework number, problem number

\begin{quote}
\item 
Let $G=(V,E)$ be an undirected graph. Unless we say otherwise, a graph has no loops or parallel edges.

\begin{itemize}
\item Prove that if $|V| \ge 2$ there are two distinct nodes $u$ and $v$ such that degree of $u$ is equal to degree of $v$.  Recall that the degree of a node $x$ is the number of edges incident to $x$.
\item Prove that if $G$ has at least one edge then there is a path between two distinct nodes $u$ and $v$ such that degree of $u$ is equal to degree of $v$.
\end{itemize}

\end{quote}
\hrule



\begin{solution}
\item 
\begin{enumerate}[(a)]
\item In a graph with $|V| \ge 2$, the number of degrees of any vertex has to be between 0 and $|V| - 1 $ because the definition of an edge requires it to exist between two vertices. This means that there are $|V|$ different options for numbers of degrees that each vertex can have. However, in the case of vertices, none of the vertices can have degree $|V| -1$ because that would force there to be no vertex with degree 0. 
This leaves $|V| - 1$ different options for number of degrees for $|V|$ different vertices. In order for no two vertices to have the same number of degrees, each of the $|V| \ge 2$ vertices has to have a different number of degrees, but since there are only $|V| - 1$ options, there have to be at least two with the same number by the pigeon-hole principle.
\item
A graph with number of edges $n \ge 1$ has, by definition of an edge, at least two vertices, and contains a connected component, which we can consider a subgraph for the remainder of this problem. Since there are $|V| \ge 2|$ vertices in the connected subgraph, the proof from the previous problem applies, confirming that at least two of the vertices in the subgraph have the same degree. Furthermore, from knowing that the subgraph is connected, we know that there exists a path between these two unique vertices with the same degree as per the definition of connected. 
\end{enumerate}
\vspace*{\fill} 
Sources (HW0, Questions 1-3): https://math.berkeley.edu/~nikhil/courses/55.s16/hw13s.pdf, Sushma Adari (sadari2), Kate Eaton (kkeaton2), Sahand Mozaffari (sahandm2)
\end{solution}




% ---------------------------------------------------------
% Change authors for all future solutions
\HomeworkHeader{0}{2}

\begin{quote}
\item The \EMPH{plus one}, $w^+$, of a string $w \in \set{\Sym0,\Sym1, \Sym2}^*$ is obtained from $w$ by replacing each symbol $a$ in $w$ by the symbol corresponding $a+1 \mod 3$. For example, $\Sym{0102101}^+ = \Sym{1210212}$.  The plus one function is formally defined as follows:
\[
	w^+ := \begin{cases}
		\e & \text{if $w=\e$}\\
		\Sym1\cdot x^+ & \text{if $w = \Sym0 x$}\\
		\Sym2\cdot x^+ & \text{if $w = \Sym1 x$} \\
		\Sym0\cdot x^+ & \text{if $w = \Sym2 x$}
	\end{cases}
\]

\begin{enumerate}
\item
{\bf Not to submit:} Prove by induction that $\abs{w} = \abs{w^+}$ for every string $w$.
\item
Prove by induction that $(x\Cdot y)^+ = x^+ \Cdot y^+$ for all strings 
$x, y \in  \set{\Sym0,\Sym1, \Sym2}^*$.
\end{enumerate}
Your proofs must be formal and self-contained, and they must invoke the \emph{formal} definitions of length $\abs{w}$, concatenation $x\Cdot y$, and 
plus one $w^+$.  Do not appeal to intuition!
\end{quote}
\hrule


\begin{solution}
\begin{quote}
\item
\textbf{Inductive Base:} $x = \epsilon$
\begin{center} $(x \Cdot y)^+ = (y)^+$ by definition of concatenation with $x = \epsilon$ 
\item $\epsilon ^+  = \epsilon$ from the definition of $\textbf{\emph{plus one}}$
\item $(x \Cdot y)^+= \epsilon ^+ \Cdot y^+ = x^+ \Cdot y^+$ 
\end{center}
\textbf{Inductive Hypothesis:}
\item $\forall$ strings $w$ with $|w| = k < n$, and all strings $y \in \Sigma^\ast$, $(w \Cdot y)^+ = w^+ \Cdot y^+$
\item
\textbf{Inductive Step:}
For string x, with $|x| = n$ greater than arbitrary $k$, $x$ can be written as $aw$, where $a \in \set{\Sym0, \Sym1, \Sym2}$ and $w$ is a string in the language $ \set{\Sym0,\Sym1, \Sym2}^*$ with $|w| =  k$ for which the inductive hypothesis holds.
\begin{center}
\item
$(x \Cdot y)^+ = ((a \Cdot w) \Cdot y)^+$ by definition of string length
\item
$((a \Cdot w) \Cdot y)^+ = (a \Cdot (w \Cdot y))^+$ by concatenation lemma
\item
$(a \Cdot (w \Cdot y))^+ = a^+ \Cdot (w \Cdot y)^+$ definition of $\textbf{\emph{plus one}}$
\item
$ a^+ \Cdot (w \Cdot y)^+ =  a^+ \Cdot (w^+ \Cdot y^+$) by inductive hypothesis
\item 
$ a^+ \Cdot (w^+ \Cdot y^+) = ( a^+ \Cdot w^+) \Cdot y^+$ by concatenation lemma
\item
$( a^+ \Cdot w^+) \Cdot y^+ = x^+ \Cdot y^+$ by declaration of $x$ and definition of $\textbf{\emph{plus one}}$
\item 
Thus, $(x \Cdot y)^+ = x^+ \Cdot y^+$.
\end{center}
\end{quote}
\vspace*{\fill} 
Sources (HW0, Questions 1-3): https://math.berkeley.edu/~nikhil/courses/55.s16/hw13s.pdf, Sushma Adari (sadari2), Kate Eaton (kkeaton2), Sahand Mozaffari (sahandm2)
\end{solution}
\hrule


% ---------------------------------------------------------

\HomeworkHeader{0}{3}

\begin{quote}
\newcommand{\bu}{\bf u}
\newcommand{\bv}{\bf v}
\item Let $\bu,\bv \in \R^2$ be two fixed vectors in the real plane.
Recursively define a set $L_n \subseteq \R^2$
as follows.
\begin{itemize}
\item $L_0 = \{\bu,\bv,{\bf 0}\}$. (${\bf 0}$ denotes the zero vector 
$(0,0)$ in $\R^2$.)
\item For integer $n > 0$, $L_n = \{ {\bf x}- {\bf y} \mid {\bf x}, {\bf y}
  \in L_{n-1} \}$.
\end{itemize}
Let $L = \bigcup_{n=0}^\infty L_n$. Also, let $D = \{ a{\bf u}+b{\bf
  v} \mid a,b \in \Z \}$ be the set of vectors obtained as integer
linear combinations of ${\bf u}$ and ${\bf v}$.
\begin{enumerate}
\item Prove that $D \subseteq L$, by giving, for each $a,b\in\Z$,
an explicit value of $n$ such that $a{\bf u}+b{\bf v}\in L_n$. 
(You don't need to
minimize the value of $n$; but you must argue why $a{\bf u}+b{\bf v} \in L_n$ for your
choice of $n$.)
\item Use mathematical induction to prove that for all integers $n\ge 0$,
$L_n \subseteq D$, and hence $L \subseteq D$.
\end{enumerate}
\end{quote}
\hrule


\begin{solution}[1]
\item
\textbf{Inductive Bases (x):} 
\item In the case $D = \{x\textbf{u} + y\textbf{v}$ $|$ $x, y \in \Z\}$
\begin{itemize}
\item For $x = 0$: $0\textbf{u} = 0$; $0 \in L_0$ by definition of $L_0$, so $\forall D| x = 0$, $x\textbf{u} \subseteq L$
\item For $x = 1$: $1\textbf{u} = \textbf{u}$; $u \in L_0$ by definition of $L_0$, so $\forall D| x = 1$, $x\textbf{u} \subseteq L$
\item For $x = -1$: $-1\textbf{u} = -\textbf{u}$; $-\textbf{u} \in L_{|-1|}$, so $\forall D| x = -1$, $x\textbf{u} \subseteq L$
\begin{itemize}
\item Since $0,$ $\textbf{u}$ are in $L_0$, $0-\textbf{u} = -\textbf{u} \in L_1$ by definition of $L_n$
\end{itemize} 
\end{itemize}
\textbf{Inductive Bases (y):} 
\begin{itemize}
\item For $x = 0$: $0\textbf{v} = 0$; $0 \in L_0$ by definition of $L_0$, so $\forall D| y = 0$, $y\textbf{v} \subseteq L$
\item For $x = 1$: $1\textbf{v} = \textbf{v}$; $u \in L_0$ by definition of $L_0$, so $\forall D| y = 1$, $y\textbf{v} \subseteq L$
\item For $x = -1$: $-1\textbf{v} = -\textbf{v}$; $-\textbf{u} \in L_{|-1|}$, so $\forall D| y = -1$, $y\textbf{v} \subseteq L$
\begin{itemize}
\item Since $0,$ $\textbf{u}$ are in $L_0$, $0-\textbf{u} = -\textbf{u} \in L_1$ by definition of $L_n$
\end{itemize} 
\end{itemize}
\textbf{Inductive Hypothesis:}
\begin{itemize}
\item $\forall k$ with $ 1 < |k| < |n|$, $\pm k\textbf{u} \in L_{|k|}$
\end{itemize}
\textbf{Inductive Step:}
\item Proving that the $x$ component of $D \subseteq L_x$ for each $x\textbf{u}$
\begin{itemize}
\item Consider $L_x$, the iteration of $L_n$ that we propose contains $x{\bf u}$. 
\begin{center} 
$L_{x-1}$ contains $\pm (x-1){\bf u}$, by the inductive hypothesis
\item $L_{x-1}$ contains $\pm{\bf u}$, by the definition of $L_n$ and the base case
\item Therefore, $L_x$ contains $\pm (x-1){\bf u} - \pm 1{\bf u} = \pm (x-1+1){\bf u} = \pm x{\bf u}$, by the definition of $L_n$
\end{center}
\end{itemize}
\item Proving that the $y$ component of $D \subseteq L_y$ for each $y\textbf{v}$
\begin{itemize}
\item Consider $L_y$, the iteration of $L_n$ that we propose contains $y{\bf v}$. 
\begin{center} 
$L_{y-1}$ contains $\pm (y-1){\bf v}$, by the inductive hypothesis
\item $L_{y-1}$ contains $\pm{\bf v}$, by the definition of $L_n$ and the base case
\item Therefore, $L_y$ contains $\pm (y-1){\bf v} - \pm 1{\bf v} = \pm (y-1+1){\bf v} = \pm y{\bf v}$, by the definition of $L_n$
\end{center}
\end{itemize}
\item Proving that, consequently, $D \subseteq L_{max(x,y) + 1}$ for all $D = \{ x{\bf u}+y{\bf
  v} \mid x,y \in \Z \}$ 
\begin{itemize}
\item Since both components have identical inductive hypotheses and indentical proofs, we can be sure that by the $max(x,y)$ iteration of the set, both the ${\bf u}$ and ${ \bf v}$ components with coefficients are present in the set. 
\item Since both terms with appropriate $\pm x, \pm y$ coefficients exist in the set $L_{max(x,y)}$, set $L_{max(x,y)}$ must contain target values $x{\bf u} - y{\bf v}$, as per definition of $L_n$. Thus, the max iteration needed to achieve any target in $D$ $(a,b) = (x,y)$ is in L beginning at iteration $L_{max(a,b) + 1}$. 
\end{itemize}
\end{solution}

\begin{solution}[2]
\item 
\textbf{Inductive Base:} Each element of $L_0$ is a subset of $D$ when:
\begin{itemize}
\item ${\bf u}$: $a = 1, b = 0$
\item ${\bf v}$: $a = 0, b = 1$
\item ${0}$: $a = 0, b = 0$
\end{itemize} 
\textbf{Inductive Hypothesis:} $\forall k \in \Z 1 < k < n, L_k \subseteq D$
\textbf{Inductive Step:} 
\item Consider $L_n$ with arbitrary $n$. In $L_{n-1}$, all elements are some value $a{\bf u} + b{\bf v}$ with $a, b \in \Z$ by the inductive hypothesis. Since $L_n$ is defined as $L_n = \{ {\bf x}- {\bf y} \mid {\bf x}, {\bf y}  \in L_{n-1} \}$, then $L_n$ must only contain values $a{\bf u} + b{\bf v}$ with $a, b \in \Z$, as integers added to integers can not result in non-integers. Therefore, $L \subseteq D$, as D is $D = \{ a{\bf u}+b{\bf v} \mid a,b \in \Z \}$.
\vspace*{\fill} 
Sources (HW0, Questions 1-3): https://math.berkeley.edu/~nikhil/courses/55.s16/hw13s.pdf, Sushma Adari (sadari2), Kate Eaton (kkeaton2), Sahand Mozaffari (sahandm2)
\end{solution}



\end{document}
