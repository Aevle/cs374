% ---------
%  Compile with "pdflatex hw0".
% --------
%!TEX TS-program = pdflatex
%!TEX encoding = UTF-8 Unicode

\documentclass[11pt]{article}
\usepackage{jeffe,handout,graphicx}
\usepackage[utf8]{inputenc}		% Allow some non-ASCII Unicode in source

%  Redefine suits
\usepackage{pifont}
\def\Spade{\text{\ding{171}}}
\def\Heart{\text{\textcolor{Red}{\ding{170}}}}
\def\Diamond{\text{\textcolor{Red}{\ding{169}}}}
\def\Club{\text{\ding{168}}}

\def\Cdot{\mathbin{\text{\normalfont \textbullet}}}
\def\Sym#1{\textbf{\texttt{\color{BrickRed}#1}}}



% =====================================================
%   Define common stuff for solution headers
% =====================================================
\Class{CS/ECE 374}
\Semester{Spring 2017}
\Authors{1}
\AuthorOne{Sushma Adari}{sadari2}
%\Section{}

% =====================================================
\begin{document}

% ---------------------------------------------------------


\HomeworkHeader{0}{1}	% homework number, problem number

\begin{quote}
\item 
Let $G=(V,E)$ be an undirected graph. Unless we say otherwise, a graph has no loops or parallel edges.

\begin{itemize}
\item Prove that if $|V| \ge 2$ there are two distinct nodes $u$ and $v$ such that degree of $u$ is equal to degree of $v$.  Recall that the degree of a node $x$ is the number of edges incident to $x$.
\item Prove that if $G$ has at least one edge then there is a path between two distinct nodes $u$ and $v$ such that degree of $u$ is equal to degree of $v$.
\end{itemize}

\end{quote}
\hrule



\begin{solution}
\item 
\begin{enumerate}[(a)]
\item The graph $G = (V,E)$ has a number of vertices $n \ge 2$, in the case where $n=2$ each distinct node $u$ and $v$ can either both have a degree of $1$ or a degree of $0$, since they can either be connected or unconnected. For every proceeding case the degree of the vertices in $G$, given $n$ nodes, can be any value between $0$ and $n-1$ since no loops or parallel edges exist in $G$. There are n options for n-1 nodes . Now consider $G$  has an arbitrary node, $A$, with degree $0$ none of the remaining nodes can have a degree of $n-1$ since it would require a shared edge with $A$. The remaining $n-1$ nodes' degrees can  be any value between 1 and n-2 so there are n-1 options for n-2 nodes. According to the pidgeon hole principle, at least $2$ of the nodes $u$ and $v$ must have the same degree in the graph $G$.
\item The graph $G = (V,E)$ has a number of edges $n$ where $n \ge 1$ and therefore by definition graph $G$ has a connected component that has at least two vertices. Applying the proof in part a to this connected component shows that there are at least two vertices in the connected componenet that have the same degree and by definition between these two vertices, since they are unique, there exists a path. 
\end{enumerate}
\end{solution}




% ---------------------------------------------------------
% Change authors for all future solutions
\HomeworkHeader{0}{2}

\begin{quote}
\item The \EMPH{plus one}, $w^+$, of a string $w \in \set{\Sym0,\Sym1, \Sym2}^*$ is obtained from $w$ by replacing each symbol $a$ in $w$ by the symbol corresponding $a+1 \mod 3$. For example, $\Sym{0102101}^+ = \Sym{1210212}$.  The plus one function is formally defined as follows:
\[
	w^+ := \begin{cases}
		\e & \text{if $w=\e$}\\
		\Sym1\cdot x^+ & \text{if $w = \Sym0 x$}\\
		\Sym2\cdot x^+ & \text{if $w = \Sym1 x$} \\
		\Sym0\cdot x^+ & \text{if $w = \Sym2 x$}
	\end{cases}
\]

\begin{enumerate}
\item
{\bf Not to submit:} Prove by induction that $\abs{w} = \abs{w^+}$ for every string $w$.
\item
Prove by induction that $(x\Cdot y)^+ = x^+ \Cdot y^+$ for all strings 
$x, y \in  \set{\Sym0,\Sym1, \Sym2}^*$.
\end{enumerate}
Your proofs must be formal and self-contained, and they must invoke the \emph{formal} definitions of length $\abs{w}$, concatenation $x\Cdot y$, and 
plus one $w^+$.  Do not appeal to intuition!
\end{quote}
\hrule


\begin{solution}
\begin{quote}
\item
\textbf{Base Case:} Let $x$ be an arbitrary string of length $0$ let $x = \epsilon$
\begin{align*}
 \item $(x\Cdot y)^+ =(\epsilon \Cdot y)^+ = (y)^+ = y^+$
\item $ = \epsilon \Cdot y^+ =  \epsilon ^+ \Cdot y^+ = x^+ \Cdot y^+$
\end{align*}
\item
\textbf{Inductive Hypothesis:}
\item $\forall$ strings $w$ with $|w| = k < n$, and all strings $y \in \Sigma^\ast$, $(w \Cdot y)^+ = w^+ \Cdot y^+$
\item
\textbf{Inductive Step:}
For string x, with $|x| = n$ greater than arbitrary $k$, $x$ can be written as $aw$, where $a \in \set{\Sym0, \Sym1, \Sym2}$ and $w$ is a string in the language $ \set{\Sym0,\Sym1, \Sym2}^*$ with $|w| =  k$ for which the inductive hypothesis holds.
\begin{center}
\item
$(x \Cdot y)^+ = ((a \Cdot w) \Cdot y)^+$ by definition of string length
\item
$((a \Cdot w) \Cdot y)^+ = (a \Cdot (w \Cdot y))^+$ by concatenation lemma
\item
$(a \Cdot (w \Cdot y))^+ = a^+ \Cdot (w \Cdot y)^+$ definition of $\textbf{\emph{plus one}}$
\item
$ a^+ \Cdot (w \Cdot y)^+ =  a^+ \Cdot (w^+ \Cdot y^+$) by inductive hypothesis
\item 
$ a^+ \Cdot (w^+ \Cdot y^+) = ( a^+ \Cdot w^+) \Cdot y^+$ by concatenation lemma
\item
$( a^+ \Cdot w^+) \Cdot y^+ = x^+ \Cdot y^+$ by declaration of $x$ and definition of $\textbf{\emph{plus one}}$
\item 
Thus, $(x \Cdot y)^+ = x^+ \Cdot y^+$.
\end{center}
\end{quote}
\end{solution}
\hrule


% ---------------------------------------------------------

\HomeworkHeader{0}{3}

\begin{quote}
\newcommand{\bu}{\bf u}
\newcommand{\bv}{\bf v}
\item Let $\bu,\bv \in \R^2$ be two fixed vectors in the real plane.
Recursively define a set $L_n \subseteq \R^2$
as follows.
\begin{itemize}
\item $L_0 = \{\bu,\bv,{\bf 0}\}$. (${\bf 0}$ denotes the zero vector 
$(0,0)$ in $\R^2$.)
\item For integer $n > 0$, $L_n = \{ {\bf x}- {\bf y} \mid {\bf x}, {\bf y}
  \in L_{n-1} \}$.
\end{itemize}
Let $L = \bigcup_{n=0}^\infty L_n$. Also, let $D = \{ a{\bf u}+b{\bf
  v} \mid a,b \in \Z \}$ be the set of vectors obtained as integer
linear combinations of ${\bf u}$ and ${\bf v}$.
\begin{enumerate}
\item Prove that $D \subseteq L$, by giving, for each $a,b\in\Z$,
an explicit value of $n$ such that $a{\bf u}+b{\bf v}\in L_n$. 
(You don't need to
minimize the value of $n$; but you must argue why $a{\bf u}+b{\bf v} \in L_n$ for your
choice of $n$.)
\item Use mathematical induction to prove that for all integers $n\ge 0$,
$L_n \subseteq D$, and hence $L \subseteq D$.
\end{enumerate}
\end{quote}
\hrule


\begin{solution}[induction]
Let $k$ be an arbitrary non-negative integer. There are several cases to consider:
\begin{itemize}
\item
Blah

\item
Snort
\begin{itemize}
\item
Squee

\item
Flub
\end{itemize}

\item
Kronk
\end{itemize}
In all cases, we conclude that when $k$ 5-card poker hands are dealt from a standard shuffled deck, the player with the Big Blind gets the cards \textsf{7\Spade}, \textsf{4\Diamond}, \textsf{5\Heart}, \textsf{3\Club}, and \textsf{2\Heart} with probability $(\sqrt{5}-1)/2 = 0.618033989$.
\end{solution}

\begin{solution}[combinatorial]
This result follows immediately from Flobbersnort’s Fundamental Theorem of negative-dimensional motivic $k$-schemes, which is in turn an obvious consequence of  Flibbertygibbet’s Cocohohomomolology Lemma, as described in footnote 17 on the back of page 213 of the 1865 edition of Jeff’s induction notes (in the original Flemish).
\end{solution}


\end{document}
